\documentclass{beamer}
\usepackage[T1]{fontenc}
\usepackage{textcomp}
\usepackage[utf8x]{inputenc}
\usepackage[british]{babel}
\usepackage{url}
\usepackage{listings}
\usepackage{graphicx}
\usepackage{semantic}
%\usepackage{multicol} 
\usepackage[labelformat=empty,tableposition=above]{caption}

\usepackage{amsmath}
\usepackage{booktabs}
%\usepackage{soul}

\usepackage{tikz}
\usetikzlibrary{shapes}
\usetikzlibrary{positioning}


%FONT
\usepackage{bera}
\usepackage[garamond]{mathdesign}
\renewcommand{\rmdefault}{ugm} % garamond
\renewcommand{\sfdefault}{ugm} % sans-serif font

% must be included after hyperref (and float)
% \usepackage[noend]{algorithmic}
% \usepackage{algorithm}
% \usepackage[noend]{algpseudocode}

\title{A finite process tree for inverse interpretation}

\author{\small Ulrik Rasmussen \and Martin Dybdal}

\institute{\textrm{Department of Computer Science, University of Copenhagen}}
\date{\today}

\mode<presentation>
{
  %\usetheme{Madrid}
  %\usetheme{Frankfurt} 
  \definecolor{uofsgreen}{rgb}{.125,.5,.25}
  \definecolor{natvidgreen}{rgb}{.196,.364,.239}
  \definecolor{kugrey}{rgb}{.4,.4,.4}
%  \usecolortheme[named=uofsgreen]{structure}
  \usecolortheme{seahorse}
  \usefonttheme[onlylarge]{structuresmallcapsserif}
  \usefonttheme[onlysmall]{structurebold}
%  \useinnertheme{circles}
  \useoutertheme{default}
}

%\logo{\includegraphics[height=1.5cm]{diku.png}}

%\setbeamersize{text margin left=1.4cm}

\usenavigationsymbolstemplate{} % fjern navigation

\setcounter{tocdepth}{1}
\renewcommand\v[1]{\boldsymbol{#1}}

\mathlig{::=}{\quad ::=\quad}
\mathlig{|->}{\mapsto}
\mathlig{<|}{\trianglelefteq}
\mathlig{+>}{\rightsquigarrow}
\mathlig{+O>}{\overset{\omega}{\rightsquigarrow}}
\mathlig{<<}{\langle}
\mathlig{>>}{\rangle}
\mathlig{<<=}{\gen}
\newcommand{\dom}{\ensuremath{{\rm dom}}}
\newcommand{\anc}{\ensuremath{{\rm anc}}}
\newcommand{\gen}{\ensuremath{~{\leq\kern-0.45em \raisebox{1pt}{$\cdot$}}~}}
%\newcommand{\gend}{\ensuremath{~{\leq\kern \raisebox{1pt}{$\cdot$}}~}}
\newcommand{\w}[1]{\ensuremath{\widehat{#1}}}


\begin{document}

\frame{\titlepage}

\begin{frame}[t, fragile]
  \frametitle{Universal resolving algorithm}
  \framesubtitle{Overview}
  \begin{description}
    \setlength{\parskip}{0.4cm}
{
  \item[Tracing] The program is traced and a \textit{process tree} is build. [covered by Ulrik]
  \item[Tabulation] Generate a possibly infinite table of input/output
    pairs from the \textit{process tree}

  \item[Inversion] The desired output is unified with each possibly
    output in the table. If unification succeeds, the corresponding
    input is returned.
}
  \end{description}
\end{frame}

\begin{frame}[t, fragile]
  \frametitle{Example program}
  I will use the following program performing addition of
  Peano-numerals, as example throughout the presentation:

\begin{verbatim}
fun Add [s, ss] y = Add ss ['S, y]
  | Add .z y = if .z == 'Z then y
               else 'error-unknown-symbol
\end{verbatim}

  \begin{center}
\tiny{
  \begin{tikzpicture}[scale=0.7]
    \tikzstyle{vtx}=[fill=black!10,minimum size=12pt,inner sep=4pt]
    \tikzstyle{leaf}=[circle, draw=black,double,inner sep=2pt]
    \node[vtx] (N2) {\texttt{Add $X_{in}$ $Y_{in}$}} [->] child
    [sibling distance=6cm, level distance=1.2cm] { node[vtx] (N3)
      {\texttt{if ...}}  child [sibling distance=3cm, level
      distance=1.5cm] { node[leaf] (N4) {\texttt{'err}} edge from
        parent { node[left] {$Xa_1\ \#\ \texttt{'Z}$} } } child
      [sibling distance=3cm, level distance=1.5cm]{ node[leaf] (N5)
        {$Y_{in}$} edge from parent { node[right] {$Xa_1 \mapsto
            \texttt{'Z}$} } } edge from parent { node[left] {$X_{in}
          \mapsto Xa_1\quad$} } } child [sibling distance=7cm, level
    distance=1.6cm]{ node[vtx] (N6) { $\begin{array}{l}
          \texttt{let $X_{in}$ = $Xe_2$;} \\
          \texttt{\ \ \ \ $Y_{in}$ = ['S, $Y_{in}$]} \\
          \texttt{in Add $X_{in}$ $Y_{in}$}
        \end{array}$
      }
      child [level distance=2cm] {
        node[vtx] (N7) {
          \texttt{Add $X_{in}$ $Y_{in}$}
        }
        edge from parent [left] {
          node[right] {$
            \begin{array}{l}
               Xe_2 \gen X_{in} \\
              \texttt{['S, $Y_{in}$]} \gen Y_{in}
            \end{array}$}
        }
      }
      edge from parent {
        node[right] {\ $X_{in} \mapsto \texttt{[}Xe_1\texttt{,} Xe_2\texttt{]} \quad$}
      }
    }
    ;
    \draw [dashed,->] (N7) .. controls +(left:3cm).. (N2);
  \end{tikzpicture}}
\end{center}
\end{frame}

\begin{frame}[t, fragile]
  \frametitle{Universal resolving algorithm}
  \framesubtitle{Tabulation} The table of input/output pairs is built
  by following each path in the process tree until a terminal node is
  reached.\vspace{4mm}

  Tabulation of \texttt{Add} using the input class:
  $$cls_{in} = <<\texttt{[$\widehat{Xa_1}$, ['S, 'Z]]},\ \emptyset>>$$

  \vspace{4mm}

  \small{
  \begin{tabular}{ll}
    Input class & Output expression \\ \hline
    $<<\texttt{['Z, ['S, 'Z]]},\ \emptyset>>$ & \texttt{['S, 'Z]}\\
    $<<\texttt{[$\widehat{Xa_1}$, ['S, 'Z]]},\ \{\widehat{Xa_1}\ \#\ \texttt{'Z}\} >>$ & \texttt{'error-unknown-symbol}\\
    $<<\texttt{[[$\widehat{Xe_2}$, 'Z], ['S, 'Z]]},\ \emptyset>>$ & \texttt{['S, ['S, 'Z]]]}\\
    $<<\texttt{[[$\widehat{Xe_2}$, $\widehat{Xa_1}$], ['S, 'Z]]},\ \{\widehat{Xa_1}\ \#\ \texttt{'Z}\}>>$ & \texttt{'error-unknown-symbol}\\
    % $<<\texttt{[[$Xe_3$, [$Xe_2$, 'Z]], ['S, 'Z]]},\ \emptyset>>$ & \texttt{['S, ['S, ['S, 'Z]]]}\\
    % $<<\texttt{[[$Xe_3$, [$Xe_2$, $Xa_1$]], ['S, 'Z]]},\ \{Xa_1\ \#\ \texttt{'Z}\}>>$ & \texttt{'error-unknown-symbol}\\
    \hspace{2cm}\vdots & \hspace{2cm}\vdots \\
  \end{tabular}}
\end{frame}

\begin{frame}[t, fragile]
  \frametitle{Universal resolving algorithm}
  \framesubtitle{Inversion}

  The following is done in the inversion step:
  \begin{itemize}
  \item The desired output is unified with each output-class in the
    table of input/output pairs.
  \item Each row where the output unifies gives rise to a
    corresponding input-class.
  \item From each input-class specific input values can be generated.
  \end{itemize}

\vspace{5mm}

If the desired output is \texttt{['S, ['S, 'Z]]}, unification only
succeeds on the third row. The corresponding input class is then
\texttt{[$Xe_1$, 'Z]}, where $Xe_1$ can take on any value.
\end{frame}

\begin{frame}[t, fragile]
  \frametitle{Flow-graph language}
  \framesubtitle{Forward interpretation}

  The closed process tree we obtain by adding generalization steps to
  the tracing algorithm, can be seen as a flow-graph program.

  \begin{center}
{\tiny
  \begin{tikzpicture}[scale=0.7]
    \tikzstyle{vtx}=[fill=black!10,minimum size=12pt,inner sep=4pt]
    \tikzstyle{leaf}=[circle, draw=black,double,inner sep=2pt]
    \node[vtx] (N2) {\texttt{Add $X_{in}$ $Y_{in}$}} [->] child
    [sibling distance=6cm, level distance=1.2cm] { node[vtx] (N3)
      {\texttt{if ...}}  child [sibling distance=3cm, level
      distance=1.5cm] { node[leaf] (N4) {\texttt{'err}} edge from
        parent { node[left] {$Xa_1\ \#\ \texttt{'Z}$} } } child
      [sibling distance=3cm, level distance=1.5cm]{ node[leaf] (N5)
        {$Y_{in}$} edge from parent { node[right] {$Xa_1 \mapsto
            \texttt{'Z}$} } } edge from parent { node[left] {$X_{in}
          \mapsto Xa_1\quad$} } } child [sibling distance=7cm, level
    distance=1.6cm]{ node[vtx] (N6) { $\begin{array}{l}
          \texttt{let $X_{in}$ = $Xe_2$;} \\
          \texttt{\ \ \ \ $Y_{in}$ = ['S, $Y_{in}$]} \\
          \texttt{in Add $X_{in}$ $Y_{in}$}
        \end{array}$
      }
      child [level distance=2cm] {
        node[vtx] (N7) {
          \texttt{Add $X_{in}$ $Y_{in}$}
        }
        edge from parent [left] {
          node[right] {$
            \begin{array}{l}
              Xe_2 \gen X_{in} \\
              \texttt{['S, $Y_{in}$]} \gen Y_{in}
            \end{array}$}
        }
      }
      edge from parent {
        node[right] {\ $X_{in} \mapsto \texttt{[}Xe_1\texttt{,} Xe_2\texttt{]} \quad$}
      }
    }
    ;
    \draw [dashed,->] (N7) .. controls +(left:3cm).. (N2);
  \end{tikzpicture}
}
\end{center}

Such closed process trees might \textit{not} be perfect. This is
simply not possible to guarantee.

\end{frame}

\begin{frame}[t, fragile]
  \frametitle{Flow-graph language}
  \framesubtitle{Conversion of process tree}

\vspace{-2mm}
  \begin{center}
\tiny
  \begin{tikzpicture}[scale=0.7]
    \tikzstyle{vtx}=[fill=black!10,minimum size=12pt,inner sep=4pt]
    \tikzstyle{leaf}=[circle, draw=black,double,inner sep=2pt]
    \node[vtx] (N2) {\texttt{Add $X_{in}$ $Y_{in}$}} [->] child
    [sibling distance=6cm, level distance=1.2cm] { node[vtx] (N3)
      {\texttt{if ...}}  child [sibling distance=3cm, level
      distance=1.5cm] { node[leaf] (N4) {\texttt{'err}} edge from
        parent { node[left] {$Xa_1\ \#\ \texttt{'Z}$} } } child
      [sibling distance=3cm, level distance=1.5cm]{ node[leaf] (N5)
        {$Y_{in}$} edge from parent { node[right] {$Xa_1 \mapsto
            \texttt{'Z}$} } } edge from parent { node[left] {$X_{in}
          \mapsto Xa_1\quad$} } } child [sibling distance=7cm, level
    distance=1.6cm]{ node[vtx] (N6) { $\begin{array}{l}
          \texttt{let $X_{in}$ = $Xe_2$;} \\
          \texttt{\ \ \ \ $Y_{in}$ = ['S, $Y_{in}$]} \\
          \texttt{in Add $X_{in}$ $Y_{in}$}
        \end{array}$
      }
      child [level distance=2cm] {
        node[vtx] (N7) {
          \texttt{Add $X_{in}$ $Y_{in}$}
        }
        edge from parent [left] {
          node[right] {$
            \begin{array}{l}
              X_{in} \gen Xe_2 \\
              Y_{in} \gen \texttt{['S, $Y_{in}$]}
            \end{array}$}
        }
      }
      edge from parent {
        node[right] {\ $X_{in} \mapsto \texttt{[}Xe_1\texttt{,} Xe_2\texttt{]} \quad$}
      }
    }
    ;
    \draw [dashed,->] (N7) .. controls +(left:3cm).. (N2);
  \end{tikzpicture}

\vspace{-2mm}
{\Large $\downarrow$ }
\vspace{1mm}

\begin{tikzpicture}[scale=0.7]
\tikzstyle{vtx}=[fill=black!10,minimum size=12pt,inner sep=4pt]
\tikzstyle{leaf}=[circle, draw=black,double,inner sep=2pt]
\node[vtx] (N2)  {\texttt{Add $X_{in}$ $Y_{in}$}} [<-]
            child [sibling distance=6cm, level distance=1.2cm] {
                node[vtx] (N3)  {\texttt{if ...}}
                    child [sibling distance=3cm, level distance=1.5cm] {
                        node[leaf] (N4)  {\texttt{'err}}
                        edge from parent {
                          node[left] {$Xa_1\ \#\ \texttt{'Z}$}
                        }
                    }
                    child [sibling distance=3cm, level distance=1.5cm]{
                        node[leaf] (N5)  {$Y_{in}$}
                        edge from parent {
                          node[right] {$Xa_1 \mapsto \texttt{'Z}$}
                        }
                    }
                    edge from parent {
                      node[left] {$X_{in} \mapsto Xa_1\quad$}
                    }
            }
            child [sibling distance=7cm, level distance=1.6cm]{
                node[vtx] (N6)  {
                  $\begin{array}{l}
                    \texttt{let $X_{in}$ = $Xe_2$;} \\
                    \texttt{\ \ \ \ $Y_{in}$ = ['S, $Y_{in}$]} \\
                    \texttt{in Add $X_{in}$ $Y_{in}$}
                  \end{array}$
                }
                child [level distance=2cm] {
                  node[vtx] (N7) {
                    \texttt{Add $X_{in}$ $Y_{in}$}
                  }
                  edge from parent [left] {
                    node[right] {$
                      \begin{array}{l}
                        X_{in} \gen Xe_2 \\
                        Y_{in} \gen \texttt{['S, $Y_{in}$]}
                      \end{array}$}
                  }
                }
                edge from parent {
                  node[right] {\ $X_{in} \mapsto \texttt{[}Xe_1\texttt{,} Xe_2\texttt{]} \quad$}
                }
    }
;

\node[circle,draw,below right=of N4] (END) {$X_{out}$};
\draw [dashed,->] (END) -- (N4);
\node at(-4.6,-3.8){$X_{out} \gen \texttt{'err}$};
\draw [dashed,->] (END) -- (N5);
\node at(-1.2,-3.8){$X_{out} \gen \texttt{$Y_{in}$}$};
\draw [dashed,<-] (N7) .. controls +(left:3cm).. (N2);
\end{tikzpicture}

\end{center}


\end{frame}


\begin{frame}[t, fragile]
  \frametitle{Backward interpretation}
  \framesubtitle{Example of inverse interpretation}

 I will now show how to evaluate the program ``backwards'', given the output \texttt{['S, 'Z]}.


  \begin{center}
{\tiny
    \begin{tikzpicture}[scale=0.7]
\tikzstyle{vtx}=[fill=black!10,minimum size=12pt,inner sep=4pt]
\tikzstyle{leaf}=[circle, draw=black,double,inner sep=2pt]
\node[vtx] (N2)  {\texttt{Add $X_{in}$ $Y_{in}$}} [<-]
            child [sibling distance=6cm, level distance=1.2cm] {
                node[vtx] (N3)  {\texttt{if ...}}
                    child [sibling distance=3cm, level distance=1.5cm] {
                        node[leaf] (N4)  {\texttt{'err}}
                        edge from parent {
                          node[left] {$Xa_1\ \#\ \texttt{'Z}$}
                        }
                    }
                    child [sibling distance=3cm, level distance=1.5cm]{
                        node[leaf] (N5)  {$Y_{in}$}
                        edge from parent {
                          node[right] {$Xa_1 \mapsto \texttt{'Z}$}
                        }
                    }
                    edge from parent {
                      node[left] {$X_{in} \mapsto Xa_1\quad$}
                    }
            }
            child [sibling distance=7cm, level distance=1.6cm]{
                node[vtx] (N6)  {
                  $\begin{array}{l}
                    \texttt{let $X_{in}$ = $Xe_2$;} \\
                    \texttt{\ \ \ \ $Y_{in}$ = ['S, $Y_{in}$]} \\
                    \texttt{in Add $X_{in}$ $Y_{in}$}
                  \end{array}$
                }
                child [level distance=2cm] {
                  node[vtx] (N7) {
                    \texttt{Add $X_{in}$ $Y_{in}$}
                  }
                  edge from parent [left] {
                    node[right] {$
                      \begin{array}{l}
                        X_{in} \gen Xe_2 \\
                        Y_{in} \gen \texttt{['S, $Y_{in}$]}
                      \end{array}$}
                  }
                }
                edge from parent {
                  node[right] {\ $X_{in} \mapsto \texttt{[}Xe_1\texttt{,} Xe_2\texttt{]} \quad$}
                }
    }
;

\node[circle,draw,below right=of N4] (END) {$X_{out}$};
\draw [dashed,->] (END) -- (N4);
\node at(-4.6,-3.8){$X_{out} \gen \texttt{'err}$};
\draw [dashed,->] (END) -- (N5);
\node at(-1.2,-3.8){$X_{out} \gen \texttt{$Y_{in}$}$};
\draw [dashed,<-] (N7) .. controls +(left:3cm).. (N2);
\end{tikzpicture}
}
  \end{center}

  As the process tree is not perfect, we will have to remember
  restrictions while performing backward interpretation. (Not
  necessary in the above example)

\end{frame}



\end{document}
